\documentclass{acm_proc_article-sp}

\usepackage{color}
\usepackage{hyperref}

\begin{document}

\title{Course-specific search engines: Semi-automated methods for identifying
  high quality topic-specific corpora} 

\numberofauthors{1}
\author{
  \alignauthor
  Neel Guha \\
  \email{neelguha@gmail.com}
}

\maketitle
\begin{abstract}
Web search is an important research tool for many high school courses. However,
generic keyword search engines have a number of problems that arise out of them
not understanding the context of search (the high school course),
leading to results that are off-topic or inappropriate as reference material. In
this paper, we introduce the concept of a course-specific search 
engine and build such a search engine for the Advanced Placement course on US
History; the results of which are evaluated by subject matter experts (high
school teachers) and judged to be overwhelmingly superior to existing search engines. We
then develop two algorithms for identifying high quality documents: one 
based on textual similarity to an authoritative source and another using structured
data data from knowledge bases. We augment both with relevance feedback and
measure their effectiveness in classifying high quality documents for
course-specific search engines. 
\end{abstract}

\section{Introduction}

Over the last decade, the World Wide Web has become one of the primary sources
of information for students. This is especially the case in high school for
subjects such as history, which often have projects that require the student to
perform independent research. 

\textcolor{red}{Spliced together description of problems from two sections. Should be
  merged to flow with paragraph above, remove any redundancy, etc. Probably fine
  to launch directly into the expanded description of the three types of
  problems---easy to process this way.}

Students typically use a search engine (such as Bing, Google or Yahoo!) to find
pages relevant to their project. Unfortunately, students encounter certain
difficulties with these keyword based search engines.  Consider several examples
of queries that might arise in the context of a course on US History. The query
[boston tea party] brings up the home page for the Boston chapter for the
political organization called the ``Tea Party''. The query [benjamin franklin]
brings up pages about Benjamin Franklin Plumbing. They also bring up pages that
are targeted at elementary school students and pages from user generated sources
such as Yahoo Answers  (which are not considered good reference material by most
teachers). Queries about places such as [gold country] are even more
problematic, bringing up results related to tourism, classifieds, etc. 

Generic search engines have a number of problems with searches that are done as
part of research on an assignment or project for an academic course. There are
three broad categories of problems, in decreasing order of severity. 

{\bf Off-Topic Results}. Often, many of the results are off topic. For example,
[benjamin franklin] brings up results about a plumbing service with that name,
[gold rush] brings up pages related to Gold country tourism, etc. The problem is
especially severe with queries involving names of places, which bring up results
about local services, real estate offerings, etc. Since students are still
learning the topic and are often conducting exploratory searches and don't know
exactly what they are looking for, they can't be expected to frame the best
query. 

{\bf Inappropriate sites}. Teachers have an expectation of research material
coming from reference sources. Web search results often include a number of
sites that don't meet this bar. Some of the different types of sites that are
not appropriate include user generated sites (such as forums and Yahoo answers),
high school sites with essays from other high schoolers, biased sites (e.g.,
ConfederateAmericanPride.com). Unfortunately, these results are interspersed
with results from much more reputable sites. The student is left to sort through
all the results. Unlike off-topic results which are mostly just a nuisance,
these kind of results can often lead the student astray. 

{\bf Wrong level}. Some of the pages returned are on topic and from reputable
sites, but are targeted at the wrong level. For example, [thomas jefferson]
returns a page from the children's version of the Library of Congress
website. More detailed queries often return papers from graduate level work. 

\textcolor{red}{Next part should talk about existing solutions, query refinement as
  well as existing academic work on topic-specific search engines. The analysis
  and discussion of existing work should probably be expanded slightly. Make
  sure to contrast specifically prior work to this approach.}

The typical solution to these problems is to construct more specific
queries. Unfortunately, this both requires an intimate knowledge of the subject,
which by definition the student does not yet have, and often eliminates
potentially good results. 

Though there is substantial work in the area of topic specific ranking in
search, much of it is focussed around exploiting the link structure of the web
to identify clusters of documents that are on the same topic. This structure
can be exploited to create topic specific Page Rank \cite{haveliwala2003topic}
for influencing ranking or to influence the order in which pages should be
crawled \cite{hsu2006topic,buntine2004scalable}. More recent work
\cite{bar2009topic} looks into topical analysis of query logs.  

 There has been very little work on the use of knowledge bases to influence the
 results shown in search. \cite{jiang2009learning} explores using ontologies for characterizing the 
user. \cite{guha2003semantic} and subsequent papers show how search results can
be augmented with snippets from a knowledge base.  

There has also been very little work on creating specialized search
engines for educational purposes. The most closely related systems are systems
like PubMed, Web of Science and Google Scholar, which are specialized 
corpura around scientific research. A description and comparison of 
these is at \cite{jacso2005google}.   

\textcolor{red}{Should revise the next two paragraphs to be more of a summary of
the next 3 sections so reader knows what to expect. Key points: course-specific
search engine for APUSH which has good results. Main ideas for two algorithms
and mention summary of the quantitative evaluation}.

Our solution is to construct a specialized search engine for
educational contexts. More specifically, we are interested in constructing a
search engine for each high school course. In this paper, we use a popular high
school course (AP US History) as our target and show how we can construct a web
search engine, that is specific to this course, that captures the richness of
the web while avoiding some of the problems associated with general purpose
search engines. We use the Google Custom search engine (CSE) platform
(\url{http://google.com/cse}) to run the search engine. With a CSE, we
specify a corpus via a set of url patterns or sites and get back a search engine
that searches only over this corpus. The CSE platform takes care of the
cumbersome tasks of crawling the web, building an index and running a search
engine. This has allowed us to not only build a search engine for AP US History
(APUSH), but to also make it widely available to students taking the course. 

The outline of the rest of the paper is as follows. We first analyze some of the
common problems encountered during searches done in the context of high school
courses such as APUSH. We then describe how we constructed and evaluated a
reference search engine for APUSH by harvesting a set of sites from Google
search and then manually curating it. While the search engine thus created is
very useful, the manual curation of thousands of sites does make it
expensive. This motivates the our algorithms for automating the construction of
course-specific corpora. 

\section{Reference search engine}

Our ultimate goal is to create a system, which, given a course textbook will
produce a search engine that is appropriate for that course. In this work, we
use [9?], which is available in PDF form, to create a search engine for the AP US
History course, which was taken by over 400,000 students in 2011 [10?]. 

Building and running a search engine requires a substantial investment in
engineering and infrastructure. Fortunately, Google offers the custom search
engine platform (CSE), which allows us to create customized search engines at a
small fraction of the cost of building a search engine from scratch. With a CSE,
we specify a subset of the web as a set of URL patterns and get a search engine
that searches only over pages that match one of these patterns. We rely on CSE
for the crawling, indexing and serving search results. Our focus is on
identifying the subset of the web that should be included in an APUSH specific
search corpus. 

In order to evaluate the relative performance of a search engine with a course
specific corpus versus one that searched the whole web, and to create a baseline
against with we could measure the automated identification of the course
specific corpus, we first create a reference search engine. 

If we knew all the queries that students will issue (in the APUSH context), a
brute force approach would be to go through the search results returned by
Google and manually pick the good results. Clearly, this is not possible both
because we don't have all the queries and because of the magnitude of manual
work that would be required. We therefore approximate this by computing likely
queries from the textbook and then, by doing the curation at the site level (as
opposed to the page level). We then put this curated list of sites into a CSE to
create our reference search engine for APUSH. 

Most history related queries include one or more proper nouns (people, events,
places, etc.). We used the digital version of the textbook [9?] to extract all
the proper nouns, using simple syntactic cues such as capitalization and
punctuation. We were able to identify 1206 distinct proper nouns, occurring a
total of 11241 times in the text. We then form queries out of tuples of
proximately occurring proper nouns and retrieve the top ten results from
Google. The 132,145 results for these queries come from 23393 sites. There are
1757 sites which appear in at least 10 results and these account for 70.2\% of
all results. We manually examined each of these sites to put them into two
buckets --- those that were bad (either off-topic (e.g., trulia.com, yelp.com)
or not appropriate for academic work (e.g., answers.yahoo.com, wikianswers.com))
and those that were good. 768 were off-topic or not appropriate, leaving us with
989 good sites. So, if we were to randomly select sites, only 56\% of the
selected sites will be good. 

We created a custom search engine using the good sites (available at
\url{http://guha.com/apushcse.html}). In order to evaluate this search engine,
we collected a set of 20 queries \textcolor{red}{Good to include queries in an
  appendix} and asked a set of 4 history teachers to
compare results from Google versus results from the custom search in a blind
side-by-side test setup. They were asked to assign one of the following ratings
to each side-by-side---side A/B is better, side A/B are about the same. The
results of this evaluation are shown in Table 1. As can be seen,  the course
specific custom search is substantially preferred.  

\textcolor{red}{Add bar chart here w/ 4+1 charts for each teacher}

\section{Methods for automation}

Our long term goal is to be able to create a course-specific search engine from
the course textbook. The reference search engine described above involved manual
curation of thousands of sites, a process that is very time consuming and
expensive. Our goal is to automatically distinguish between sites that produce
off-topic results / sites that are inappropriate for academic usage and sites
that may be used for academic work.  Rather than attempting a binary
classification of sites into Good and Bad, we rank sites based on the likelihood
of them returning good results for APUSH searches. A CSE can then be configured
to include just the top N sites or better, to prefer sites that are ranked
higher. 

\textcolor{red}{Links should be made here about topic-specific
  PageRank. Essentially this is a course-specific quality score.}

The root of the problem is that a two or three word query does not communicate
the context in which the student is trying to use the search engine, i.e., the
APUSH course.  When an off-topic site (such as Benjamin Franklin Plumbing) comes
up for a history query (such as [benjamin franklin]), the problem is that the
query is inherently ambiguous and there is no way to communicate the context of
the search (i.e., the APUSH course) to the search engine.  Our overall approach
is to use this larger context in an offline process to construct the right
collection of sites. 

We now describe a progression of algorithms that can be combined to improve on
this baseline. We start with a classical information retrieval style analysis of
the content of search results page and of the APUSH textbook. Sites that are
more similar to the APUSH textbook are more likely to yield better search
results. In the next step, we use a knowledge base of facts about the real world
entities to improve the ranking obtained by just textual similarity. Finally, we
present a relevance feedback based learning algorithm that improves the search
engine based on user interaction with search results. 

\subsection{Authoritative documents}

\textcolor{red}{Instead of writing this section in terms of an explanation of
  TF-IDF and vector space models which are classic methods, I would rephrase this
  as an alternative way to compute a topic-specific quality score using an
  authoritative reference which is a new idea.}

Classical information retrieval (IR), one of the primary mechanisms used by
search engines such as Google, is based on the notion of similarity between two
pieces of text, typically, the query and the web page.  In our case, we compute
the similarity between the pages on a site and the APUSH textbook. Sites that
are more similar to the APUSH textbook should  be more likely to return better
results. 

One of the most commonly used similarity measures is the cosine similarity
metric based on the vector space model of documents \cite{salton1975vector}. In
this model, each document is a vector in an N dimensional where each term in the
corpus is a dimension. The coordinates of a document along the ith axis is given by by a
product of the term's frequency in the document (TF) and the a measure of how
significant that term is in the corpus. The later is usually measured by Inverse
Document Frequency (IDF), which is the log of the inverse of the number of
documents in the corpus that the term appears in. The similarity between two
documents in this model is the cosine of the angle between the vectors between
the two documents. 

Aiand Bi are the tf-idf for each term i in the corpus, for the two documents. We
compute the similarity of a site to the APUSH textbook as follows. We retrieved
the content of 110,121 of the 132,145 pages returned by Google as results for
the queries that we used to create the reference search engine (many pages could
not be retrieved either because the sites blocked our robot or because the site
was down). After stripping the pages of all the HTML markup and javascript, we
stem the words on each page (using the Porter Stemmer
\cite{porter1980algorithm}) and extract the terms from (the pages from) each
site along with their frequency of occurrence. We also stem all the terms in the
APUSH textbook. We then compute a similarity score between each site and the
APUSH textbook using the cosine similarity metric.

While textual similarity based metrics are very powerful for identifying
off-topic sites, they are not capable of identifying sites that are on-topic,
but likely inappropriate for academic purposes (such as answers.yahoo.com or
ConfederateAmericanPride.com). These sites are indeed on-topic, but pure
similarity based metrics cannot detect the bias. For this, we need some form of
human judgement. We next describe a mechanism for getting this by analyzing the
search engine's usage. 

\subsection{Knowledge based approach}

Most queries and web documents are about real world entities.  This realization
has lead many of the big search engines to build various kinds of knowledge
bases to augment their search results \textcolor{red}{Nice to cite blog post
  here}. There has also been a lot of work in the area of the semantic web
and linked data \cite{berners2001semantic,bizer2008linked} which aim to build a
large distributed network of information about entities and the relations
between them. In this section, we describe how this source of information can be
used to improve the ranking based on simple textual similarity.  

Some types of entities lead to more off-topic results than others. For example,
places (e.g., Virginia, Maryland) are more likely to lead to results that are
not about history compared to US presidents since the former will bring up real
estate and local results. US Presidents in turn are more likely to bring up
off-topic results compared to Confederate Generals, since the former are more
likely to have institutions and places named after them. This relative preference
(Confederate Generals better than US Presidents better than Places) captures
some of the APUSH context. 

Our goal is to automatically identify the types that are more likely to give
good results and give greater preference to the sites that are pulled up for
queries that contain references to these kinds of entities. Before we can do
that, we need to map the proper nouns we extract from the text to entities and
the entities to types of entities. Ideally, we would like to do this in a
fashion that is not specific to history, but will work with little or no change
to many other subjects. To do this, we need a broad knowledge base about a large
number of entities, along with information about the type of each
entity. Wikipedia contains this kind of information about a large number of
entities and DBPedia makes this information available as a structured knowledge
base. Each ``thing'' in Wikipedia corresponds to an entity in DBPedia
(www.dbpedia.org) and each ``category'' in Wikipedia corresponds to a DBPedia
type. We will use DBPedia as the primary source of entities, types and for
mapping proper nouns to entities and entities to types. 

There could be many different proper nouns corresponding to a single real world
entity. For example, President Lincoln, Abraham Lincoln, Abe Lincoln, etc. all
refer to the same person. Similarly, a given proper noun (e.g., Washington) could
map to multiple different entities. We found that the most robust way of mapping
from proper nouns to candidate entities is to use search itself. For every
proper noun P, we issue the query [P site:wikipedia.org]. Each of the search
urls has the form http://en.wikipedia.org/wiki/<entity-id>. The first three entities
for each query are used as candidate entities for the corresponding proper
noun. Each entity may have a number of categories associated with it. For
example, the entity (whose unique identifier is) Abraham\_Lincoln has 29
different categories associated with it (e.g., American\_Presidents,
Illinois\_Lawyers, Assasinated\_HeadsOfState, etc.). This mapping is done using
the RDF dumps from DBPedia. 

The next step is to assign a score to each category. The score is a measure of
the likelihood of a query containing a proper term that refers to an entity in
that category bringing up a site with on-topic results. The intuition behind the
scoring algorithm is as follows. A course, such as APUSH, is about certain
categories of entities and the relationships between them. These categories
should be assigned higher scores. An example of such a category would be
American\_Presidents. There will be a number of other categories that also
appear, but only incidentally. These should get lower scores. Examples of such
categories would be Illinois\_Lawyers and Noble\_titles. We would expect that a
larger fraction of the entities in a category that the course is about will
occur in the textbook compared to categories that appear incidentally. The score
for the $i$th category is given by
\begin{equation}
CategoryScore_i = \frac{\#Textbook_i}{\#DBpedia_i}
\end{equation}
where $\#Textbook_i$ and $\#DBpedia_i$ count occurences of entities in the
textbook and DBPedia, respectively. 

For example, the textbook contains references to 33 entities in the category
American\_Presidents, which, in DBPedia is associated with 44 entities, giving
this category a score of 0.75. On the other hand, even though the text contains
references to more Harvard\_University\_Alumni (34), a total of 6533 entities in
DBPedia are associated with this category, giving Harvard\_University\_Alumni a
much lower score than American\_Presidents. 

\textcolor{red}{Put Table 4/5 in a single two-column table here in latex format}

Table 4 gives some of the top rated
categories for APUSH, i.e., the categories that are considered most relevant to
APUSH. As can be seen, of the hundreds of thousands of categories in
DBPedia (which includes categories for rock stars, planets, etc.), the
ones that come up are indeed very apropos to US History.           
We then score each entity by summing the scores for the categories that it is
associated with. For example, the entity Abraham\_Lincoln gets a contribution
from each of the 29 categories that it is a part of. Table 5 gives some of the
top rated entities. 

\textcolor{red}{Table 6 here}

Next, we score each proper term by adding the scores of all the entities that it
could refer to. So, since ``Abraham Lincoln'' could refer to the president or the
movie with that name, each entity contributes a score. Table 6 gives some of the
top rated proper terms. Again, as can be seen, of the millions of entries in
DBPedia, the ones chosen are indeed very highly apropos to the APUSH context. 

We then score each query based on the proper terms in the query. Finally we
score each of the sites based on the scores associated with the queries for
which they produced results. The score for the site is the average of the query
scores. This gives us a ranking of sites by the likelihood of them being a good
candidate for inclusion into the APUSH search engine. 


\textcolor{red}{We should tie this approach (and the topicality scoring
  approach) directly to the the 3 problems laid out in the introduction:
  off-topic, inappropriate, wrong level. In particular, 
  its clear that both deal with off-topic results easily but not so clear how
  they work for inappropriate and wrong level.}

\subsection{Relevance feedback}

Using relevance feedback to improve the performance of systems is an established
technique \cite{salton1997improving}. As the search engine gets used, we can
interpret clicks from the user as feedback. For example, if users repeatedly
skip results from a certain site (even when pages from that site are ranked
higher), preferring results from certain other sites, they are expressing a
judgement about the appropriateness of that site for APUSH. 

\textcolor{red}{First, describe how we use the labeled data to generate fake
  relevance feedback. It would be best to use the equation environment to make
  it easy to parse. }

Evaluating relevance feedback based algorithms requires large amounts of usage
data, which we don't have. So, we instead reused some of the manually curated
sites to evaluate the utility of relevance feedback in our context. Of the sites
that had been manually curated, we randomly selected 50 good sites and 50 bad
sites. In practice, this list of good and bad sites would be obtained from usage
logs. We aggregate the good sites and bad sites, each, into a single composite
document, so that we have 2 large documents, one corresponding to the good sites
(GD) and another corresponding to the bad sites (BD). Then, for each site we are
trying to score, we compute the similarity of the site to GD and BD. The net
relevance feedback score is the Good Score minus the Bad Score. 

\textcolor{red}{Now, with the fake feedback data out of the way we can describe
  the two methods in a way that is agnostic as to whether the data is fake
  agnostic as to whether its real or fake data}. 

The overall score for the site is the score for similarity to the APUSH textbook + the score
for similarity to the good sites - score for similarity to the bad sites. 

We can also incorporate relevance feedback to incrementally improve
performance. As with relevance feedback for similarity scoring, we sample 50
good and bad sites from the manually curated list to simulate data from actual
usage of the search engine. We compute a relevance feedback score for each
category as follows. We take the categories associated with each query (from the
previous step) and propagate them to the sites associated with the query to get
a set of categories associated with each site. The score associated with each
category is (Ng-Nb)where Ngis the number of good sites associated with the
category and Nbis the number of bad sites associated with the category. Then, as
before, we propagate the category scores to the sites.  

\section{Experimental results}

Of the 1453 sites we examined as part of the manual curation, 582 were good, 871
tended to give off-topic or inappropriate (i.e., bad) results. So, randomly
picking sites for inclusion into the search engine will lead to 40.05\% of the
sites being good. This forms the baseline for evaluating algorithms for
automating the curation. 

Each of our algorithms produces a list of sites ranked according to their
likelihood of producing good results for APUSH queries. We use the manually
curated list of good sites to measure how well an algorithm performs. The ideal
algorithm would rank each of the 989 good sites above the bad sites. Therefore,
a good metric for evaluating an algorithm is the number of good sites included
in the top 100, 200, 300, 400 and 500 sites picked by the algorithm. 

\textcolor{red}{More details of experimental methodology here and discussion
  of results from previous sections and put here. Also include the description
  of the simple frequency-based scoring approach here as this is really a baseline
  not a proposed method. Include graph.}  

\textcolor{red}{Include graphs of results here 

\subsection{Relevance feedback and hybrid scoring}

\textcolor{red}{Include graphs and discussion of results here when you add in
  relevance feedback and hybrid scoring}

As a final step, we investigate combining the statistical, similarity based
score with the knowledge based score. We compute a combined score by simply
adding the similarity score and the knowledge based score. Table 8 gives the
results from this hybrid approach. As can be seen, there is a very small
improvement over just the similarity score. We believe that there is room for
improvement here and this is one of the directions of future work. 


\bibliographystyle{plain}
\bibliography{apush_cse}

\end{document}
