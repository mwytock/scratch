\documentclass[12pt]{article}

\usepackage{amssymb}
\usepackage{amsmath}
\usepackage{epsfig, graphics}
\usepackage{latexsym}
\usepackage{fullpage}
\usepackage[parfill]{parskip}
\usepackage[tight]{subfigure}
\usepackage{hyperref}

\addtolength{\textheight}{.3in}

\title{Sustainable energy with Machine learning}


\DeclareMathOperator*{\minimize}{\mathrm{minimize}}

\author{Matt Wytock}
\pagestyle{empty}
\begin{document}

Satisfying energy demand by harnessing sustainable resources is one of the most important challenges facing our nation and world. In this proposal, I consider the renewable integration problem and propose solutions based on large-scale probabilistic models and optimization-based control. I believe that developing these state-of-the-art machine learning methods to exploit the vast amounts of information now available with the smart grid will be fundamental to integrating our sustainable resources effectively.

The transformation to sustainable resources is already underway; for example, installed wind capacity has increased 4-fold in the past 5 years \cite{doe.12}. However, integrating renewable energy sources into the electrical grid presents fundamental challenges to the way we approach planning and provisioning. In particular, unlike traditional fossil fuels, solar and wind energy is highly variable and non-dispatchable, creating significant problems for operators who must meet demand reliably. Uncertainty also arises due to the difficulty in modeling demand, which although well-studied, remains a open problem. Typically, operators compensate by provisioning slack capacity using forecasting methods. However, the vast majority of existing methods either focus on making point estimates or predict just the marginal distribution for a single location. In reality, these distributions are highly correlated across both time and space; by treating each prediction independently, we lose critical information.

This spatiotemporal forecasting task requires jointly modeling many locations at multiple future time points, resulting in a difficult high-dimensional inference problem beyond the scope of classical methods. However, methods from machine learning excel at modeling high-dimensional distributions using millions of data points; this is exactly the scenario that now arises with the smart grid. Thus, I propose developing new large-scale probabilistic models for spatiotemporal forecasting and new optimization-based control techniques for using these models in energy planning. By taking a probabilistic approach, we can accurately model real-world correlations which will significantly improve our energy planning capabilities.

First, in my own ongoing work, started in September of this year \cite{wytock.12}, I consider spatiotemporal forecasting and propose a new model, sparse Gaussian conditional random fields. Concretely, consider the task of forecasting wind power; let $y \in \mathbb{R}^p$ represent wind power forecasts across multiple locations and times and let $x \in \mathbb{R}^n$ represent input features such as past wind power and weather forecasts. The model has a simple log-linear form
\begin{equation}
p(y|x;\Lambda,\Theta) \propto \exp \left\{ -\frac{1}{2}y^T\Lambda y - x^T\Theta y \right\}
\end{equation}
where the quadratic term, parameterized by $\Lambda$, captures spatiotemporal correlations between wind power forecasts and the linear term, parameterized by $\Theta$, captures the dependence of future wind power on past wind power and weather forecasts. The power of this model comes from the fact that sparse parameters (few nonzero entries) can represent complex correlations between many variables while being tractable to learn from data. This fits perfectly with spatiotemporal forecasting task, where we want to learn correlations across many different times and locations from millions of data points. In the language of machine learning, the model combines the benefits of sparse inverse covariance selection \cite{banerjee.08} and $\ell_1$-regularized regression. In this work,  I also formulate parameter estimation in this model as a convex optimization problem and develop a novel algorithm to find the solution. Finally, I demonstrate that the model achieves state-of-the-art results for forecasting energy demand in the Pennsylvania power grid on a small dataset.

Although this model is compelling for spatiotemporal forecasting, in order to apply it to a wide variety of smart grid problems, I must address the core challenge of scaling it to real world problem sizes. For example, in the case of wind forecasting, we may have tens if not hundreds of nearby wind farms for which we want to forecast wind power and years' worth of data from which to learn. Distributed convex optimization is an area of active research in machine learning; for example, Alternating Direction Method of Multipliers techniques \cite{boyd.11} provide a solution for decomposing problems across the feature space under certain conditions. By addressing these difficult questions and scaling the sparse Gaussian conditional random field model, I will develop a broadly useful state-of-the-art probabilistic model for spatiotemporal forecasting, which will dramatically improve our ability to quantify uncertainty in many tasks relevant to renewable integration and energy planning. Furthermore, it is important to note that although this model was developed specifically for problems relating to sustainable energy, this research will also advance the state-of-the-art in machine learning, as it is generally applicable to high-dimensional multivariate regression problems.

Given a high-dimensional probabilistic forecasting model, my next step is to develop control methods that accurately exploit this information to make energy planning decisions. The existing state-of-the-art grid control methods, such as model predictive control, are not well-suited to this task because they are ``certainty equivalent,'' simply taking the mean prediction as input for their optimization. Instead, I will develop an approach based on Monte Carlo methods whereby we sample potential future scenarios from our probabilistic model and plan using multiple runs of model predictive control. This approach naturally handles the spatiotemporal correlations in our forecasting models and adapts accordingly.

By developing state-of-the-art machine learning methods for spatiotemporal forecasting and probabilistic model based control, we can significantly improve the ability of the smart grid to incorporate renewable energy sources. Fundamentally, these methods rely on large datasets of the kind that are recently available due to smart meters and sensor networks in the smart grid. By unlocking the power of this data with the methods described in this proposal, we can accelerate the transformation toward sustainable energy with tremendous impact for our country and the entire world.

\renewcommand{\refname}{\vskip -1.5cm}
\footnotesize
\bibliographystyle{plain}
\bibliography{proposed}

\end{document}
