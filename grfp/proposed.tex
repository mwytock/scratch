\documentclass[12pt]{article}

\usepackage{amssymb}
\usepackage{amsmath}
\usepackage{epsfig, graphics}
\usepackage{latexsym}
\usepackage{fullpage}
\usepackage[parfill]{parskip}
\usepackage[tight]{subfigure}
\usepackage{hyperref}

\title{Sustainable energy with Machine learning}

\DeclareMathOperator*{\minimize}{\mathrm{minimize}}

\author{Matt Wytock}
\begin{document}

Satisfying energy demand by harnessing sustainable resources is one of the most important challenges facing our nation and world. In this proposal, I consider problems arising from integrating renewable energy sources into the smart grid and propose solutions based on large-scale probabilistic models and optimization-based control. I believe that developing these state-of-the-art machine learning methods to exploit the vast amounts of information now available with the smart grid will be fundamental in utilizing our sustainable resources effectively.

The transformation toward sustainable resources is already underway, for example installed wind capacity has increased 4-fold in the past 5 years [?]. However, integrating renewable energy sources into the electrical grid presents fundamental challenges to the way we approach energy planning and provisioning. In particular, solar and wind energy is inherently more variable than traditional fossil fuels, which creates significant problems for operators who must meet energy demands reliably. These  problems are excacerbated by the difficulty in modeling energy demand, which although well-studied, remains a open problem. Typically, operators compensate by provisioning extra slack capacity which increases the cost of deploying renewable energy sources, thus restricting their adoption.

A better alternative would be to improve the capability of the smart grid to more effectively compensate for natural variations in supply and demand. Modeling energy demand and weather factors which drive renewable energy production are inherently difficult problems; both are examples of complex systems that we cannot describe with a concise set of equations. However, with the deployment of the smart grid and related technologies such as smart meters, we have the opportunity to collect massive amounts of data. Developing the computational methods necessary to utilize this information is the critical issue which will define effectiveness of the smart grid. Here, I propose large-scale probabilistic models for forecasting energy supply and demand, as well as new optimization-based control techniques to utilize these models for energy planning.

First, I consider spatiotemporal forecasting tasks in which we make predictions across multiple nearby locations at multiple future time points. Examples of such takss that arise in energy planning include forecasting energy loads at multiple points in the grid or forecasting wind at multiple nearby wind farms. In these scenarios, understanding correlations across multiple locations is critical to quantifying uncertainty in order to make decisions about power systems. However, the vast majority of existing methods either focus on making point estimates or predict just the marginal distribution for a single location. In part, this is due to the difficult high-dimensional infererence problem that arises from jointly modeling correlations across both space and time; a problem which is beyond the scope of classical methods.

In our own recent work started in September of this year \cite{wytock.12}, we consider this forecasting scenario and propose a new model, sparse Gaussian conditional random fields, in which we represent forecasting as a multivariate regression problem. Concretely, consider the task of forecasting wind and let $y \in \mathbb{R}^p$ be a vector output variables representing future wind forecasts across multiple locations and times and let $x \in \mathbb{R}^n$ be our input features. Our model has a simple log-linear form
\begin{equation}
p(y|x;\Lambda,\Theta) \propto \exp \left\{ -\frac{1}{2}y^T\Lambda y - x^T\Theta y \right\}
\end{equation}
where the quadratic term, parameterized by $\Lambda$, captures spatiotemporal dependencies between the wind forecasts and the linear term, parameterized by $\Theta$, models the dependency of these forecasts on external factors. In this previous work, we also formulate parameter estimation as a convex optimization problem which we solve by developing a novel algorithm. It is important to note that although this model was developed specifically for problems relating to sustainable energy (and achieves state-of-the-art results for energy demand forecasting), it also advances the state-of-the-art in machine learning, as it is generally applicable to high-dimensional multivariate regression problems.

However, despite these advances, there is significant work remaining to make this model truly useful for solving real-world problems. First, to date we have applied the model only to the problem of forecasting energy demand in Pennsylvania and thus there is significant work to consider other problems. More challenging is scaling the model to real-world problem sizes; the current algorithm is $O((n+p)^3)$ and cannot be distributed across multiple machines. However, by applying and adapting recent advances in convex optimization, for example Alternating Direction Method of Multipliers (ADMM) techniques [?], I believe we can solve these problems and develop a state-of-the-art method for spatiotemporal forecasting tasks.

Given a high-dimensional probabilistic forecasting model, the next step is to develop control methods that accurately exploit this information to make energy planning decisions. The existing state-of-the-art grid control methods, such as model predictive control, are not well-suited to this task because they are ``certainty equivalent,'' simply taking the mean prediction as input for their optimization. Instead, I propose utilizing Monte Carlo methods whereby we sample potential future scenarios from our probabilistic model and plan using multiple runs of model predictive control. This approach naturally handles the spatiotemporal correlations in our forecasting models and adapts accordingly.

By developing state-of-the-art machine learning methods for spatiotemporal forecasting and probabilistic model based control, we can significantly improve the ability of the smart grid to incorporate renewable energy sources. Fundamentally, these methods rely on the availability of large-scale data of the kind that is recently available due to smart meters and sensor networks. By unlocking the power of this data with the methods described in this proposal, we can accelerate the transformation toward sustainable energy with tremendous impact for our country and the world as a whole.

\bibliographystyle{plain}
\bibliography{proposed}

\end{document}
