\documentclass[12pt]{article}

\usepackage{amssymb}
\usepackage{amsmath}
\usepackage{epsfig, graphics}
\usepackage{latexsym}
\usepackage{fullpage}
\usepackage[parfill]{parskip}
\usepackage[tight]{subfigure}
\usepackage{hyperref}

\addtolength{\textheight}{.3in}
\pagestyle{empty}

\title{Personal statement}

\author{Matt Wytock}
\begin{document}

Now is a fantastic time to be a computer scientist. Due to the rapid pace at which computational methods continue to transform every aspect of the modern world, we have tremendous opportunity for broad impact across a wide variety of domains. I believe that this opportunity also comes with a social responsibility to utilize this power to improve the greater good and that we have many opportunities to do exactly that. In this essay, I briefly describe my own personal journey as a computer scientist, why I am excited about the potential for machine learning to have a big impact on critical problems in the field of sustainable energy and finally my future career aspirations and goals.

Personally, I have always taken a hands-on approach to computer science. During my career as an undergraduate at UCSD, I held a variety of part-time programming jobs, including working with a meteorologist modeling climate change at the Scripps Institute of Oceanograpy and working at Neurome, a small neuroscience startup founded by Floyd Bloom, former editor-in-chief of Science magazine. My experience at Scripps underscored the importance and ubiquity of computational methods in the sciences -- as a first-year student, I built an automated pipeline for processing large NASA satellite datasets which was critical to research efforts. At Neurome, we developed new technologies for visualizing and analyzing large ($>$50 GB) volumetric data with the goal of better automating the way neuroscience research is done. Although both were great learning experiences, it was clear in both environments that the emphasis was on scientific goals, not in pushing the state-of-the-art in computation.

For this reason, I was excited to take a position at Google upon graduation in 2005. Here, I had the opportunity to work on many challenging problems at web-scale and learn firsthand various state-of-the-art technologies such as Bigtable and MapReduce. I also gained invaluable teamwork and leadership experience; as my career progressed, I went from an individual contributor to the technical lead of a team with more than 10 people. Along the way, we also launched many successful projects \cite{claiborne,gultekin,mayer,seth}, one of which I will go into more detail in my previous research essay. Google was a tremendous learning experience and I believe it was the best route for me immediately after college. Beyond the technical knowledge acquired, I also came to realize that what I enjoyed most was working on early-stage, research-oriented projects with the strongest team available. Perhaps most importantly, at Google I was also exposed to the impact of large-scale machine learning, which deeply influenced my desire to pursue research more broadly. 

A mentor at Google once framed this desire with the following audacious question: ``We will achieve human-level A.I. in the next N years, what role do you want to play?'' I don't recall the precise value of N  and besides, it is impossible to judge the accuracy of this claim; it is well-known that similar statements have proven false in the past. Nonetheless, the broad impact of machine learning on large datasets is already undeniable. There are many such well-known efforts at Google including spelling correction, speech recognition and autonomous vehicle navigation. However, I feel strongly that I will be able to make the largest possible contribution to the field by developing the state-of-the-art through a career in academia.

Thus, while still working full-time at Google, I began enrolling in graduate courses at Stanford as a first step in getting involved in research. Through these courses I began to learn about the current state-of-the-art and what quickly became clear was that in many different domains, the best computational results are currently achieved by applying a common set of principles and methods that form the basis of machine learning.

Upon leaving Google and entering CMU in 2010, I was excited to have the opportunity to work on research full-time. However, I had not yet formulated a clear direction and, in particular, was undecided about whether to focus on the theoretical or more applied aspects of machine learning research. On one hand, machine learning is compelling due to a core set of theoretically sound methods that can be applied across many problems; yet, fundamentally I have a desire to solve tangible real world problems. 

It is my belief that the challenges posed by sustainable energy are some of the most urgent and important facing the scientific community and the world. I also believe that with the advent of the smart grid and the opportunity to collect large amounts of data, many of these challenges are fundamentally computational and will require advancing the new methods in machine learning, as I discuss in my research proposal. Thus, this research fulfills the dual goals of having broad real world impact as well as advancing the theoretical state-of-the-art. However, even more importantly, I strongly believe that by developing sustainable energy resources we will be fulfilling a basic social good in saving our planet from the disastrous effects of climate change. This is a monumental challenge but it is exactly the sort that we should be tackling as computer scientists.

Ultimately, making progress will require advancing the state-of-the-art across many disciplines including Computer Science, Electrical Engineering, Mechanical Engineering as well as working together with partners in industry. I believe that I am well-positioned to do this at CMU and upon graduation, I plan to continue pursuing this research in a faculty position.

\renewcommand{\refname}{\vskip -1.5cm}
\footnotesize
\bibliographystyle{plain}
\bibliography{personal}


\end{document}
